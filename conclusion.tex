\section{Conclusion}
\label{s:conclusion}

We performed the first long term controlled evaluation of how client-side anti-phishing defenses - including blacklists, content-based, and real-time - affect the new phishing websites' detection and blacklisting coverage and speed in the most popular web browser by launching and monitoring a large set of phishing sites. 

َAccording to the record volume of phishing attack \cite{APWG}, these defense mechanisms detect several phishing websites. Nevertheless, Our research uncovered a significant fact that Chrome's client-side anti-phishing without reporting fails in detecting new phishing websites. In addition, our study revealed an essential inconsistency between Mobile and Desktop blacklists. 

Given the advancement of sophisticated phishing attacks, we believe that analyzing major browsers anti-phishing system is essential to identify the weaknesses, improve the browsers' defense, and protect the billions of users from attackers. In addition, our study provide clear picture of users' privacy in the browser client-side anti-phishing system.

Analysis of data from our empirical measurement, we believe that longitudinal measurements of the modern browsers' anti-phishing ecosystem are crucial not only for maintaining a comprehension
of the modern browsers' protection, but also for
evaluating new security features as they are released to prove that the security of users can be continuously ensured.