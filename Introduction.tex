\section{Introduction}
\label{s:intro}

Phishing attacks remain popular among criminals with the growth of sophisticated phishing websites and the aid of evasion techniques such as cloaking~\cite{oest2020phishtime,hao2013understanding,internet-crime-complaint-center-(ic3)-business-e-mail-compromise-the-12-billion-dollar-scam}. 
With the support of illicit underground services, these fraudulent sites cause collateral damage by harming the reputation of impersonated brands, compromising legitimate infrastructure, and demands effort to assuage abuse~\cite{zhang2007cantina}.

There are two categories of approaches in protecting users against phishing attacks: reactive and proactive methods. Blacklists, as a reactive approach, require blacklists entities to validate the fraudulency of the visited website. Content-based methods as a proactive approach analyze the content of the website~\cite{wang2011verilogo}.

Blacklists have shown high accuracy (low false-positive rates) in identifying phishing websites. However, there is a noticeable window of risk between the time the phishing site first goes up and the time it is available in the browsers' blacklists. Moreover, Blacklists also not efficient in detecting \textit{targeted phishing attacks} such as \textit{spear phishing} attacks which target a specific group of people~\cite{burns2019spear}.
Thus, the content-based anti-phishing defense method comes to the rescue to not only validate the visited website proactively to identify likely phishing pages but also to detect the small phishing campaign.

Nowadays, the modern desktop and mobile web browsers utilize a content-based anti-phishing system besides the default anti-phishing blacklists to protect the user against known and unknown malicious sites. The primary purpose of appending this mechanism to the browsers is to cover the shortcomings of blacklists in detecting new malicious sites~\cite{googlechromeprivacywhitepaper}. Still, there is a need to evaluate these client-side classifiers to assess their performance, especially when facing phishing attacks with evasion techniques. Moreover, this thorough evaluation will assist in increasing the security of the users in the browsers with the client-side anti-phishing system.

In this paper, we study the effectiveness of the client-side anti-phishing system, including the content-based, Real-time, and password protection mechanism deployed in Chrome, as the most popular browser in the world. We focus on addressing the client-side anti-phishing shortcomings of Google Safe Browsing~(GSB) since it has the largest share of users worldwide, with around three billion active users~\cite{statcounterall}.
For this purpose, we propose an empirical analysis contributing to the Phishfarm framework~\cite{oest2019phishfarm}
 
First, we extract the Chrome browsers' built-in classifier, Google's phishing pages filter (GPPF), via the static and dynamic analysis on chromium implementation. 
The extracted classifier information includes the classification algorithm, the number of scoring rules, the corresponding rules weight, scores formula, and the classifier's features. 
Based on classifier information, we evaluate the client-side and server-side classifier. Our experiments reveal that only 12\%  of the phishing websites have scored more than the threshold, and the GSB server only detects 15\%  of the phishing websites among the websites that the client-side classifies as phishing websites. 

Our experiment's central part, done on a large scale, is measuring the content-based anti-phishing effects on the blacklist's speed and coverage.
Nowadays, many phishing kits employ server-side cloaking, and they make the presence of cloaking a  norm within modern phishing websites~\cite{oest2018inside}.
To perform a real-world study, we considered server-side cloaking with an IP address filter as a common evasion technique in desktop and mobile browsers.
We consider four different anti-phishing components that the chrome browser uses: (1) blacklist, (2) Content-based, (3) real-time, and (4) password protection. This browser utilizes these anti-phishing mechanisms in three different security layers: (1) no protection, (2) standard protection, and (3) enhanced protection.

We perform this experiment for seven days and monitor the blacklisting process to see the effects of Chrome's different defense layers in cloaking and reporting to the main entities.
The experiment results lead to several interesting findings: 
despite GSB claims, the client-side content-based anti-phishing does not affect the blacklisting speed and accuracy.

Moreover, the chrome browser's real-time anti-phishing system has serious shortcomings, and the server classifier results in the real-time anti-phishing requests are inefficient. 
In terms of privacy, in the client-side anti-phishing ecosystem, clients sending a request to the server to run the server-side classifier and analyze the site after detecting a malicious site. The browser sends some privacy-related information in this request.
The password protection approach is also vulnerable to manipulating DOM elements names. 
There is also a considerable inconsistency in the desktop and mobile blacklists. Even in the absence of cloaking, and after reporting the phishing sites to the blacklists entities, mobile chrome blacklists providers do not blacklist reported phishing sites.

The contributions of our paper are as follows:

\begin{itemize}
    \item The first in-depth and continuous long-term measurements of chrome browsers' anti-phishing, including content-based, real-time, and password protection methods, and disclosure of these three components' weaknesses and inefficiency.
    \item Identification and disclosure of remaining significant inconsistency in the desktop and mobile blacklists.
    \item The first empirical study evaluates the GSB server-side classifier performance dealing with phishing websites in the wild.
    \item Disclosure of serious privacy issues in the chrome browser's client-side content-based anti-phishing system
\end{itemize}
